\documentclass[12pt]{amsart}
\usepackage[margin=1in]{geometry}
\usepackage{paralist}

\theoremstyle{definition}
\newtheorem{question}{Question}

\begin{document}
\begin{center}
\textbf{\Huge
Lesson Plan: Meeting Thirteen
}
\end{center}
\vspace{.5in}

\section*{Phase One}

Do Peer Instruction about rationals versus reals.

\begin{question}
What is the decimal notation for the rational number $8/15$?
\end{question}

\begin{question}
Let $y$ be the real number with decimal notation $y = 0.12111111\ldots$ (the 1's repeat forever). Write $y$ in the form of a rational number $a/b$.
\end{question}

\begin{question}
Since $\sqrt{17}$ is irrational, the decimal notation goes on forever without becoming a repeating pattern of any length. T/F
\end{question}

\begin{question}
Since $11/17$ is a rational number, its decimal notation eventually settles into a repeating block of digits. Without any work, you can be sure that this repeating block is no longer than
\begin{compactitem}
\item 1 digit
\item 11 digits
\item 16 digits
\item 17 digits
\end{compactitem}
\end{question}

\section*{Phase Two}

More time to discuss different matching schemes.
Clean up older, unfinished problems.
Add newer problems.

See Meeting13.tex.


\end{document}