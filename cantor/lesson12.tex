\documentclass[12pt]{amsart}
\usepackage[margin=1in]{geometry}
\usepackage{paralist}

\theoremstyle{definition}
\newtheorem{question}{Question}

\begin{document}
\begin{center}
\textbf{\Huge
Lesson Plan: Meeting Twelve
}
\end{center}
\vspace{.5in}

\section*{Phase One}
Do Peer Instruction about properties of the real numbers as described with the decimal system.

\begin{question}
Divide the interval from $0$ to $1$ into 100 equal pieces. Suppose that $x$ lies in piece \# 84 from the left. What is the most you can say about the decimal notation for $x$?
\begin{compactitem}
\item $0.83\ldots$
\item $0.84\ldots$
\item $0.85\ldots$
\item $0.083\ldots$
\item $0.084\ldots$
\end{compactitem}
\end{question}

\begin{question}
Divide the interval from $0$ to $1$ into 1000 equal pieces. The point $x$ lies in subinterval number 10 from the left. What can you say about the beginning of the decimal notation for $x$?
\begin{compactitem}
\item $0.10\ldots$
\item $0.09\ldots$
\item $0.090\ldots$
\item $0.009\ldots$
\item $0.010\ldots$
\end{compactitem}
\end{question}

\begin{question}
Consider a real number whose decimal notation starts $x = 0.433\ldots$. Where is $x$ in the interval from $0$ to $1$?
\begin{compactitem}
\item In the fifth subinterval of length $1/10$ from the left.
\item In the fourth subinterval of length $1/10$ from the left.
\end{compactitem}
\end{question}

\section*{Phase Two}

More time to discuss different matching schemes.
Clean up older, unfinished problems.
Add newer problems.

See Meeting12.tex.



\end{document}