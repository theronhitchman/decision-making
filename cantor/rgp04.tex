\documentclass[12pt,letterpaper]{article}

\usepackage[utf8]{inputenc}
\usepackage[T1]{fontenc}
\usepackage{amsmath}
\usepackage{amsfonts}
\usepackage{amssymb}
\usepackage{amsthm}
\usepackage[left=2cm,right=2cm,top=2cm,bottom=2cm,headheight=22pt]{geometry}
\usepackage{fancyhdr}
\usepackage{setspace}
\usepackage{lastpage}
\usepackage{graphicx}
\usepackage{caption}
\usepackage{subcaption}
\usepackage{paralist}
\usepackage{url}

\theoremstyle{definition}
\newtheorem{question}{Question}
\newtheorem{example}{Example}
\newtheorem{exercise}[question]{Exercise}
\newtheorem*{challenge}{Challenge}
\newtheorem*{theorem}{Theorem}
\newtheorem*{definition}{Definition}

\begin{document}

%Paramètres de mise en forme des paragraphes selon les normes françaises
\setlength{\parskip}{1ex plus 0.5ex minus 0.2ex}
\setlength{\parindent}{0pt}

%Paramètres relatifs aux en-têtes et pieds de page.
\pagestyle{fancy}
\lhead{Hitchman}
\chead{\Large Cantor: Reading and Guided Practice \#4}
\rhead{Spring 2016}
\lfoot{\emph{Mathematics in Decision Making}}
\cfoot{}
\rfoot{\emph{\thepage\ of \pageref{LastPage}}}

\section*{Introduction}
This assignment is a further discussion of the idea of a matching.
We examine other common pitfalls encountered when making matchings, and how matchings can be used to make comparisons.

\section*{Goals}
At the end of this assignment, a student should be able to:
\begin{compactitem}
\item Describe the idea of a matching clearly in plain language.
\item Describe two common mistakes made when making matchings.
\item Use the idea of a matching to make comparisons between the sizes of sets.
\item Construct matchings between sets.
\end{compactitem}


\section*{Reading and Questions for Cantor's Paradise Meeting 5}

Let us continue our examination of the idea of a matching between the elements of two sets.

\subsection*{The Definition of a Matching.}
In the last reading, we learned an official definition of a matching between the elements of two sets.
It looks like this:
\begin{definition}
Let $A$ and $B$ be two sets.
A \emph{matching between the elements of $A$ and the elements of $B$} is a way to associate to each element of $A$ exactly one element of $B$, so that no elements of either set are left unpaired.
\end{definition}

A tricky thing here is that the matching is \emph{the association}.
Another way to describe this is that a matching is a \emph{rule for assignment}: for each element $a$ of the set $A$, the rule tells us how to find an particular element $b$ of the set $B$.
These two elements $a$ and $b$ are then paired together.
The part of the definition which says that for each member $a$ of the set $A$ there must be exactly one associated member $b$ of $B$, and vice versa, expresses the idea of a \emph{one to one correspondence.}

\begin{example}
Let $T = \{ 1, 4, 7, 10, 13, \ldots, 3001\}$.
Let $J = \{ 2, 5, 8, 11, 14, \ldots, 3002\}$.
We now describe a matching between the elements of $T$ and the elements of $J$.

Given an element of $T$, we can associate to it the element of $J$ which is one greater. 
If you want, you can describe this with algebraic symbols like this: 
\begin{quote}
To each element $x$ of $T$ we associate the element $x+1$ of $J$.
\end{quote}
But, really, all we are doing is specifying the following rule for assignment. If you give me an element of $T$, I find the associated element of $J$ by \textbf{adding one}.

Note that this rule is completely reversible.
If we start with an element of $J$, we can find the associated element of $T$ by \textbf{subtracting one}.
This is the way that a matching between elements is just a rule for assignment.
We give the instructions for how to find each elment's pair-partner in the other set.
\end{example}


\begin{exercise}
Explain in sentences how you can be sure that this matching between the elements of $T$ and the elements of $J$ are in a one to one correspondence.
\end{exercise}

\subsection*{Common Mistakes}

Think about counting. 
Counting is a form of making a matching between those things you wish to count and some subset of the natural numbers.
In particular, to count a collection of things is the same as making a matching of those things with some ``initial subset'' of $\mathbb{N}$. 
When counting a baby's toes (certainly one of the most fun things you can do), you are really making a matching between the set $\{ \text{toes on this baby} \}$ and the set $\{1, 2, 3, 4, 5, 6, 7, 8, 9, 10\}$.

In the last reading, we described a common error in the business of making matchings.
\begin{example}
My friend Penny is very young, and is still learning to count.
When she first tried it, she often counted things like this:
\begin{quote}
One, Three, Four, Five, Eight, \dots
\end{quote}
Penny's mistake is that she is skipping over some numbers. 
Thus the construction of a matching between what ever she is counting and the set of natural numbers is faulty.
There will be relevant natural numbers which don't have pair-partners! 
Essentially, nothing gets labeled "two," so the matching is doomed.
\end{example}

\begin{exercise}
My little friend Penny no longer makes that mistake of skipping numbers.
Instead, she makes a different mistake.
Now she counts like this:
\begin{quote}
One, two, three, four, five, six, seven, eight, nine, ten, eleven, eight, nine, ten, eleven, eight, nine, ten, eleven, \dots
\end{quote}
(It can go on like this for a long time. You get the idea.)
What is the nature of Penny's second mistake?

Use the language of matchings between elements to describe what is going wrong.
\end{exercise}

\subsection*{Making Comparisons}

An important feature of having a matching between the elements of one set and the elements of another set is that it means that in a strict sense those two sets have the same size. 
This is nice, because it gives us a way to say things are "the same" directly, without having to go through the intermediate step of counting those sets. 
(Counting really large sets is often inconvenient.)


Often, the ways in which we fail when attempting to make matchings can give us clues to the relative sizes of two sets.
The precise way in which we cannot produce a one to one correspondence might tell us which of the two sets is bigger!

For example, In class we saw that the set $C$ of chairs in the classroom had more elements than the set $S$ of students. 
We observed this because having everyone sit down was a form of attempting to make a matching between the elements of $S$ and the elements of $C$. 
Each student (an element of $S$) associated himself or herself with a chair (an element of $C$) by sitting in it. 
That is a rule for the assignment!

\begin{exercise}
Which of two of Penny's mistakes are made in trying to form this matching?
\end{exercise}

\begin{exercise}
Suppose that somehow our assignment of students to chairs failed to be a matching because it made the other mistake Penny taught us about.
What would be happening, physically, with students and chairs?
\end{exercise}

\begin{exercise}
Suppose that we have two sets where $A$ is definitely larger than $B$.
Why is it impossible to make a matching by coming up with a rule of association that starts with elements of $A$ and produces elements of $B$?
\end{exercise}

\begin{exercise}
Suppose that we have two sets where $A$ is definitely larger than $B$.
Why is it impossible to make a matching by coming up with a rule of association that starts with elements of $B$ and produces elements of $A$?
\end{exercise}



%\begin{thebibliography}{9}
%\end{thebibliography}

\end{document}
%sagemathcloud={"zoom_width":100}