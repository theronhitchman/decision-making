\documentclass[12pt]{amsart}
\usepackage[margin=1in]{geometry}

\theoremstyle{definition}
\newtheorem{task}{Task}

\begin{document}

\begin{center}
\textbf{\Huge
Cantor's Paradise: Class Meeting \#9
}
\end{center}
\vspace{.25in}
Student Responses from Spring 2015
\vspace{.25in}

\begin{task}
What is meant by this phrase?
\begin{quote}
``$x$ is an element of the set $Y$''
\end{quote}
\end{task}

\begin{center}
\textit{
$Y$ is a grouping of numbers or symbols, and $x$ is one of these numbers or symbols.
}
\end{center}
\vspace{.25in}

\begin{task}
What is meant by this phrase?
\begin{quote}
``There is a matching between the elements of $X$ and the elements of $Y$.''
\end{quote}
\end{task}

\begin{center}
\textit{
For every individual in the group named $X$, there is an individual and unique partner in the group named $Y$, and there is no element left unpaired in either group.
}
\end{center}
\vspace{.25in}

\begin{task}
How is counting like making a list? How is it like matching?
\end{task}

\begin{center}
\textit{
Counting is making a matching between your set and an initial segment of the natural numbers.}\\

\textit{This is like making a list beacuse you are putting the elements in a specific order and that order holds a matching with an initial segment of $\mathbb{N}$.
}
\end{center}
\vspace{.25in}

\begin{task}
Describe a matching between the elements of $\mathbb{N}$ and the elements of the \emph{integers}:
\[
\mathbb{Z} = \{ \ldots, -3, -2, -1, 0, 1, 2, 3, \ldots \}
\]
\end{task}

\begin{center}
\textit{ We make a list out of the elements of $\mathbb{Z}$ with the following pattern:
\[
0, 1, -1, 2, -2, 3, -3, 4, -4, \ldots
\]
The ordering here describes a matching with the elements of $\mathbb{N}$, since the first element is matched with $1$, the second element is matched with $2$, and so on.
}
\end{center}

\end{document}