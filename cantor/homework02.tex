%% Fall 2013 MDM Homework Template
\documentclass[12pt,letterpaper]{article}

\usepackage[utf8]{inputenc}
\usepackage[T1]{fontenc}
\usepackage{amsmath}
\usepackage{amsfonts}
\usepackage{amssymb}
\usepackage[left=2cm,right=2cm,top=2cm,bottom=2cm,headheight=22pt]{geometry}
\usepackage{fancyhdr}
\usepackage{setspace}
\usepackage{lastpage}
\usepackage{graphicx,subcaption}

\begin{document}

%other parameters
\setlength{\parskip}{1ex plus 0.5ex minus 0.2ex}
\setlength{\parindent}{0pt}

%header and footer parameters
\pagestyle{fancy}
\lhead{Math 1100}
\chead{Weekly Homework}
\rhead{Due: March 4}
\lfoot{}
\cfoot{\emph{Prof. Hitchman}}
\rfoot{}

\begin{center}
{
\Large
\textbf{Cantor: Assignment \#2}
}
\end{center}

One way to represent numbers is with \emph{dot pictures}.
Here are some different dot pictures for the number 10:
\[
\begin{array}{l}
OOOOO\\ OOOOO
\end{array}
\qquad
\begin{array}{l} OOO\\ OOO \\ OOO \\ O\end{array}
\qquad\begin{array} {l} OOO\\ OO\\ OO\\ O\\ O\\ O\end{array}
\qquad\begin{array}{l} OOOOOOO\\ O \\ O \\ O\end{array}
\qquad\begin{array}{l} OOO\\ OOO\\ OO\\ OO\end{array}
%\qquad
%\begin{array}{l}OO\\ OO\\ OO\\ O\\ O\\ O\\ O\end{array}
\]

The definition of the term \emph{dot picture} requires exactly that there are as many dots as the given number, and each row has at least as many dots as the row immediately below it.

Let $T$ be the set of dot pictures that represent the number 10 and have exactly three rows.

Let $J$ be the set of ways to represent 10 as a sum of three positive integers.
For example, here are some elements of the set $J$:
\begin{equation*}
\begin{array}{cccc}
10 & = 5 &+ 3 &+ 2 \\
10 & = 4 &+ 3 &+ 3\\
10 & = 6 & +2 &+ 2 \\
\end{array}
\quad
\begin{array}{cccc}
10 & = 8 &+ 1 &+ 1 \\
10 & = 6 &+ 3 &+ 1\\
10 & = 7 & +2 &+ 1 \\
\end{array}
\end{equation*}
Notice that for this task we consider two representations as being the same if all you do is reorder the numbers.
So $10 = 5 + 3 +2$ is the same thing as $10 = 2 + 3 + 5$ and also the same thing as $10 = 2 + 5 + 3$.
Also, the representation $10 = 9 + 1 + 0$ is not an element of $J$ because $0$ is not a positive integer.

In one written page, describe a matching between the elements of $T$ and the elements of $J$.
Discuss why this means that these two sets have the same number of elements.




\end{document}
%sagemathcloud={"zoom_width":100}