\documentclass[12pt]{amsart}
\usepackage[margin=1in]{geometry}
\usepackage{cancel}

\theoremstyle{definition}
\newtheorem{task}{Task}
\newtheorem*{untask}{Task}

\begin{document}

\begin{center}
\textbf{\Huge
Cantor's Paradise: Class Meeting \#11
}
\end{center}


\vspace{.5in}

Today, we study the set $\mathbb{Q}^+$ of all the positive rational numbers.

\begin{task}
Consider this table of fractions. Write a few sentences to explain why every element of $\mathbb{Q}^+$ appears on this table at least once.
\begin{table}[htdp]
\begin{center}
\begin{tabular}{|c|c|c|c|c|c|c|}
\hline
1/1 & 2/1 & 3/1 & 4/1& 5/1 & 6/1& \dots \\
\hline
1/2 & 2/2 & 3/2 & 4/2 & 5/2 & 6/2& \dots \\
\hline
1/3 & 2/3 & 3/3 & 4/3 & 5/3 & 6/3& \dots \\
\hline
1/4 & 2/4 & 3/4 & 4/4 & 5/4 & 6/4& \dots \\
\hline
1/5 & 2/5 & 3/5 & 4/5 & 5/5 & 6/5& \dots \\
\hline
\vdots & \vdots & \vdots & \vdots & \vdots & \vdots & $\ddots$ \\
\hline
\end{tabular}
\end{center}
\label{default}
\caption{An Array of Positive Rational Numbers}
\end{table}%
\end{task}

\vspace{2in}

\begin{task} Now consider this list of numbers.
\[
\frac{1}{1}, \frac{2}{1}, \frac{1}{2}, \frac{3}{1}, \frac{2}{2}, \frac{1}{3}, \frac{4}{1}, \frac{3}{2}, \frac{2}{3}, \frac{1}{4}, \frac{5}{1}, \frac{4}{2}, \frac{3}{3}, \frac{2}{4}, \frac{1}{5}, \frac{6}{1}, \dots
\]
Write a few sentences to explain why every element of $\mathbb{Q}^+$ appears in this list at least once.\\
\end{task}

\clearpage

\begin{task}
Now we thin this sequence out by keeping only fractions in lowest terms and throwing out the others
\[
\frac{1}{1}, \frac{2}{1}, \frac{1}{2}, \frac{3}{1}, \cancel{\frac{2}{2}}, \frac{1}{3}, \frac{4}{1}, \frac{3}{2}, \frac{2}{3}, \frac{1}{4}, \frac{5}{1}, \cancel{\frac{4}{2}}, \cancel{\frac{3}{3}}, \cancel{\frac{2}{4}}, \frac{1}{5}, \frac{6}{1}, \dots
\]
to get
\[
\frac{1}{1}, \frac{2}{1}, \frac{1}{2}, \frac{3}{1}, \frac{1}{3}, \frac{4}{1}, \frac{3}{2}, \frac{2}{3}, \frac{1}{4}, \frac{5}{1}, \frac{1}{5}, \frac{6}{1}, \dots
\]
Write a few sentences to explain why every element of $\mathbb{Q}^+$ appears exactly once in the new list.\\
\end{task}

\vspace{2in}

\begin{task}
Write a few sentences to explain why $\mathbb{Q}^+$ is countably infinite.
\end{task}

\vspace{2in}

\begin{task}
Now explain why the set $\mathbb{Q}$ of all rational numbers is countable infinite.
\end{task}

\vspace{1.5in}

\begin{untask}
Contemplate how seriously weird that last result is.
\end{untask}




\end{document}