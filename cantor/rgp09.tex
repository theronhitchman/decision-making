\documentclass[12pt,letterpaper]{article}

\usepackage[utf8]{inputenc}
\usepackage[T1]{fontenc}
\usepackage{amsmath}
\usepackage{amsfonts}
\usepackage{amssymb}
\usepackage{amsthm}
\usepackage[left=2cm,right=2cm,top=2cm,bottom=2cm,headheight=22pt]{geometry}
\usepackage{fancyhdr}
\usepackage{setspace}
\usepackage{lastpage}
\usepackage{graphicx}
\usepackage{caption}
\usepackage{subcaption}
\usepackage{paralist}
\usepackage{url}

\theoremstyle{definition}
\newtheorem{question}{Question}
\newtheorem{example}{Example}
\newtheorem{exercise}[question]{Exercise}
\newtheorem*{challenge}{Challenge}
\newtheorem*{theorem}{Theorem}
\newtheorem*{definition}{Definition}

\begin{document}

%Paramètres de mise en forme des paragraphes selon les normes françaises
\setlength{\parskip}{1ex plus 0.5ex minus 0.2ex}
\setlength{\parindent}{0pt}

%Paramètres relatifs aux en-têtes et pieds de page.
\pagestyle{fancy}
\lhead{Hitchman}
\chead{\Large Cantor: Reading and Guided Practice \#9}
\rhead{Spring 2015}
\lfoot{\emph{Mathematics in Decision Making}}
\cfoot{}
\rfoot{\emph{\thepage\ of \pageref{LastPage}}}

\section*{Introduction}
We reacquaint ourselves with old friends, the \emph{rational numbers}.

\section*{Goals}
At the end of this assignment, a student should be able to:
\begin{compactitem}
\item Describe clearly what a rational number is.
\item Recognize when two rational numbers are equal.
\item Recognize when one rational number is greater than another.
\item Discuss the ordering of rational numbers, and how it differs from the ordering of the natural numbers.
\end{compactitem}
A student might also be able to:
\begin{compactitem}
\item Show that the set of rational numbers is an infinite set.
\end{compactitem}

\section*{Reading and Questions for Cantor's Paradise Meeting 10}

Recall that we have encountered the set of \emph{natural numbers}
\[
\mathbb{N} = \{1, 2, 3, 4, \ldots\}
\]
and the set of \emph{integers}
\[
\mathbb{Z} = \{ \ldots, -3, -2, -1, 0, 1, 2, 3, \ldots\}.
\]
Now we will introduce a new kind of number and consider the set of all of them.

A \emph{rational number} is one of the form $a/b$ where $a$ is any integer and $b$ is any natural number. 
For example, $-4/7$ comes from choosing $a = -7$ from $\mathbb{Z}$ and $b= 7$ from $\mathbb{N}$.

\begin{exercise}
Make three more examples of rational numbers and say what the choices of $a$ and $b$ are.
\end{exercise}

The collection of all rational numbers is denoted $\mathbb{Q}$. 
(Here the funny Q is for ``quotient,'' since we make rational numbers by making a quotient of $a$ by $b$.)

The first interesting thing to work out about rational numbers is that some of them are equal without it being immediately obvious!
For example, $1/2$ is really the same thing as $2/4$.
Typically, these are called \emph{equivalent} rational numbers. 

\begin{definition} Two rational numbers $a/b$ and $a'/b'$ are \emph{equivalent} exactly when 
$ab' = a'b$.
\end{definition}
Note the multiplication is just ordinary multiplication of integers like in grade school.


\begin{exercise}
List three pairs of equivalent rational numbers.
\end{exercise}

\begin{exercise}
List three pairs of inequivalent numbers.
\end{exercise}

\subsection*{Comparing Rational Numbers}

One nice thing about the natural numbers is that we can arrange them in a line.
Like an infinite line of telephone poles along the highway, they march off away from you, going on forever.
And between each natural number and the next is a gap.
In fact, that gap is what makes it meaningful to say ``the next natural number.''

Things are more challening for the rational numbers.
First, can we even order them on a line?
Yes. 
Like so.

\begin{definition}
We say that the rational number $a/b$ is \emph{larger} than the rational number $c/d$ when 
\[
a \cdot d > c \cdot b.
\]
\end{definition}

\begin{exercise}
Find an example of rational numbers $a/b$ and $c/d$ where $a/b$ is larger than $c/d$.
\end{exercise}

\begin{exercise}
Find an example of rational numbers $a/b$ and $c/d$ where $a/b$ is smaller than $c/d$.
\end{exercise}

\begin{exercise}
Place all of the rational numbers you have found so far on a number line showing their relative sizes.
\end{exercise}

\begin{exercise}
Explain why the following is true:
\begin{quote}
For each pair of rational numbers $a/b$ and $c/d$, either $a/b$ is larger than $c/d$, or $a/b$ is equivalent to $c/d$, or $a/b$ is smaller than $c/d$.
\end{quote}
\end{exercise}

\subsection*{Density of the Rational Numbers}

So we see that both the rational numbers and the natural numbers can be placed along a line.
How are things really different for the rational numbers?
Well, they keep popping up in different places.

\begin{exercise}
Find a rational number which is larger than $1/2$ and smaller than $3/4$.
\end{exercise}


\begin{exercise}
Find a rational number which is larger than $1/2$ and smaller than $6/11$.
\end{exercise}

\begin{exercise}
Find a rational number which is larger than $5/51$ and smaller than $93/175$.
\end{exercise}

\begin{exercise}
Suppose you are given a pair of rational numbers $a/b$ and $c/d$.
That is, you know they are rational numbers, but you do not know exactly what they are.
We assume that $a/b$ is smaller than $c/d$. 

Show how to find a rational number $x/y$ so that $x/y$ is larger than $a/b$ and smaller than $c/d$.
\end{exercise}

\subsection*{Challenges}

\begin{challenge}
Use the definition of \emph{infinite set} to prove that the set $\mathbb{Q}$ of rational numbers is an infinite set.
\end{challenge}

\begin{challenge}
Repeat the last exercise, but use a \emph{different subset} of $\mathbb{Q}$ to help you do the job.
\end{challenge}



%\begin{thebibliography}{9}
%\end{thebibliography}

\end{document}
%sagemathcloud={"zoom_width":100}