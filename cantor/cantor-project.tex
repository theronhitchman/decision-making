\documentclass[10pt]{amsart}
\usepackage[margin=1in]{geometry}

\theoremstyle{definition}
\newtheorem{task}{Task}

\begin{document}

\begin{center}
\textbf{\Huge
Cantor's Paradise: Unit Capstone Project
}
\end{center}

\section*{Specifications for Grading}

To earn credit, this assignment must
\begin{itemize}
\item be typed, of no more than one page in length, though pictures may be hand-drawn; 
\item answer the questions below clearly and correctly;
\item conform to reasonable standards for grammar, spelling, and usage of the English language with minimal errors. (You may consider seeking help on writing from the Writing Center in the Academic Learning Center. http://www.uni.edu/unialc/writing-center);
\item be turned in by 2pm (the end of class) on Friday, April 10.
\end{itemize}


\section{The Questions}


We have spent some time thinking about different representations of real numbers.
So far, we have encountered \emph{decimal} (base $10$) notation and \emph{binary} (base $2$) notation for elements of the set $\mathcal{R}$.

Recall that $\mathcal{R}$ is the set of points on some line segment with ends $O$ and $I$. We will set the length of the segement $\mathcal{R}$ to be equal to $1$ unit.

\begin{task}
Now we will use a \emph{ternary} (base three) system. Instead of dividing each piece into ten identical pieces, divide into identical thirds, which you can think of as ``left,'' ``center,'' and ``right.''
We use the symbols $l$, $c$, and $r$ for this system.

Locate the intervals represented by the symbols $l$, $c$, $r$, $ll$, $lc$, $lr$, $cl$, $cc$, $cr$, $rl$, $rc$, $rl$, $rrlc$ and $lrlrccr$.
\end{task}

\begin{task}
Working now with countably infinite strings, describe how to locate a real number that is described by mathematical word consisting of the symbols $l$, $c$ and $r$.\\
\end{task}

\begin{task}
By constructing a matching, show that $\mathcal{R}$ has the same size as the set $\mathcal{I}$ of all countably infinite mathematical words consisting of the letters $l$, $c$ and $r$.\\
\end{task}

In a class meeting, we named the set of all elements of $\mathcal{I}$ which do not contain the letter $c$ to be $\mathcal{I}_c$, and we found a matching between the elements of $\mathcal{C}$ and the elements of $\mathcal{I}_c$.
So, somehow, these two sets ``have the same number of elements.'' 
Now we want to investigate the length of the set $\mathcal{I}_c$.\\

\begin{task}
Draw a representation of the set of all real numbers represented in ternary $l$-$c$-$r$ notation that do not have $c$ as the first letter.
How long is this set?\\
\end{task}

\begin{task}
Draw a representation of the set of all real numbers represented in ternary $l$-$c$-$r$ notation that do not have $c$ as the first or second letter. How long is this set? \\
\end{task}

\begin{task}
Draw a representation of the set of all real numbers represented in ternary $l$-$c$-$r$ notation that do not have $c$ as the first, second, or third letter.
How long is this set? \\
\end{task}

\begin{task}
Draw a representation of the set $\mathcal{I}_c$ of all real numbers whose ternary representation does not use the letter $c$. \\
\end{task}

\begin{task}
What can you say about the length of the set $\mathcal{I}_c$?
\end{task}

\begin{task}
Consider all of the different matching we have between the sets $\mathcal{R}$, $\mathcal{C}$, $\mathcal{I}$, and $\mathcal{I}_c$. Then consider their lengths. What weird thing is happening?
\end{task}

Cool, huh? This makes my head hurt sometimes.

\end{document}
%sagemathcloud={"zoom_width":100}