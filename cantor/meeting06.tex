\documentclass[12pt]{amsart}
\usepackage[margin=1in]{geometry}

\theoremstyle{definition}
\newtheorem{task}{Task}
\newtheorem{challenge}[task]{Challenge}
\newtheorem*{definition}{Definition}

\begin{document}

\begin{center}
\textbf{\Huge
Cantor's Paradise: Class Meeting \#6
}
\end{center}


\vspace{.5in}

\begin{challenge}
Horror of horrors, you are in charge of every third grader in Cedar Falls for an afternoon.
Fortunately, you have them all trapped in the UNI-Dome.
They are all running around on the field shouting, ``Look at me! I'm a vampire unicorn!"

Quick, without counting, come up with a way to decide if there are more boys, more girls, or the same number of each.
\end{challenge}

\texttt{We have discussed this, but be sure you are comfortable with it. \\
Talk it over now. How does the idea of a matching work here?}

\vspace{.1in}
\hrule
\vspace{.1in}

So far, we have shown that a couple of different sets have the same size as the set of natural numbers $\mathbb{N} = \{1, 2, 3, 4, 5, \dots\}$.
This happens a lot, so we want some language to describe it.

\begin{definition}
We say that a set $X$ is \emph{countably infinite} when there is a matching between the elements of $X$ and the elements of $\mathbb{N}$.
\end{definition}

\begin{task} Let $\mathcal{O}$ be the set of odd natural numbers,
$\mathcal{O} = \{ 1, 3, 5, 7, 9, 11, 13, 15, 17, \dots\}$.
Discuss why $\mathcal{O}$ is countably infinite.
\end{task}

\vspace{.1in}
\hrule
\vspace{.1in}

In the reading for today, you encountered the notion of an \emph{initial subset} of the natural numbers.
For example, $\lfloor 5 \rfloor = \{1, 2, 3, 4, 5\}$.

Also, we have seen that the \emph{power set} $\mathcal{P}(X)$ of a set $X$ is the set of all subsets of $X$.

\begin{definition}
Fix some number $n$. A \emph{$0/1$ sequence of length $n$} is a string of $n$ symbols, each of which is either a $0$ or a $1$.
For example, $00100111$ is a $0/1$ sequence of length $8$.
\end{definition}

\begin{task}
Let $B_5$ be the set of all $0/1$ sequences of length $5$.
Find a matching between the elements of $B_5$ and the elements of $\mathcal{P}(\lfloor 5 \rfloor)$.
\end{task}

\begin{task}
Let $B_{73}$ be the set of all $0/1$ sequences of length $73$.
Find a matching between the elements of $B_{73}$ and the elements of $\mathcal{P}(\lfloor 73 \rfloor)$.
\end{task}


\end{document}
%sagemathcloud={"zoom_width":100}