\documentclass[12pt]{amsart}
\usepackage[margin=1in]{geometry}
\usepackage{paralist}

\theoremstyle{definition}
\newtheorem{question}{Question}

\begin{document}
\begin{center}
\textbf{\Huge
Lesson Plan: Meeting Four
}
\end{center}
\vspace{.5in}


This is a group work day.
Work in groups of three or four: it is a day to mix them up!
Give them a variety of pairs of sets to compare.
They will need plenty of assistance on the following:
\begin{compactitem}
\item Checking carefully what counts as an element of each set.
Have them make examples and non-examples.

\item connecting the idea of comparing sizes to the idea of a matching.
\end{compactitem}

\vspace{.25in}

Be sure to stop and sum up after the first question.
It will likely take only two minutes for everyone to get that far.
Use the opportunity to connect the language of a matching between the elements.
(What are the two sets? What are there elements? What is the attempted matching? Where is the problem with it? Note that this is the same as Penny's mistake!)

\vspace{.25in}

Sum up again after another ten or fifteen minutes, and again with only ten minutes remaining.
Put student work at the forefront.
Help them connect language.

\vspace{.25in}

Be sure to point out how weird the last one is.


\end{document}
%sagemathcloud={"zoom_width":100}