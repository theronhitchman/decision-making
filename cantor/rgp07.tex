\documentclass[12pt,letterpaper]{article}

\usepackage[utf8]{inputenc}
\usepackage[T1]{fontenc}
\usepackage{amsmath}
\usepackage{amsfonts}
\usepackage{amssymb}
\usepackage{amsthm}
\usepackage[left=2cm,right=2cm,top=2cm,bottom=2cm,headheight=22pt]{geometry}
\usepackage{fancyhdr}
\usepackage{setspace}
\usepackage{lastpage}
\usepackage{graphicx}
\usepackage{caption}
\usepackage{subcaption}
\usepackage{paralist}

\theoremstyle{definition}
\newtheorem{question}{Question}
\newtheorem{example}{Example}
\newtheorem{exercise}[question]{Exercise}
\newtheorem*{challenge}{Challenge}

\begin{document}

%Paramètres de mise en forme des paragraphes selon les normes françaises
\setlength{\parskip}{1ex plus 0.5ex minus 0.2ex}
\setlength{\parindent}{0pt}

%Paramètres relatifs aux en-têtes et pieds de page.
\pagestyle{fancy}
\lhead{Hitchman}
\chead{\Large Cantor: Reading and Guided Practice \#7}
\rhead{Spring 2015}
\lfoot{\emph{Math and Decision Making}}
\cfoot{}
\rfoot{\emph{\thepage\ of \pageref{LastPage}}}

\section*{Introduction}

Here, you pause to take stock and think the context and structure of what you have learned.

\section*{Goals}
At the end of this assignment, a student should be able to:
\begin{compactitem}
\item Make an outline of the topics we have studied so far.
\item Describe how our study of sets, matching, and counting adds up.
\end{compactitem}


\section*{Reading and Questions for Cantor Meeting 8}


It is time to reflect a little bit. After three weeks, you have learned a lot.

\begin{exercise}
Reread the previous \textsc{Reading and Guided Practice} assignments.
\end{exercise}

\begin{exercise}
Make an outline of what you have learned so far.
How are the topics related?
What are the major points of which to keep track?
Have you discovered any new ways of thinking you should note for later?
\end{exercise}


\end{document}
%sagemathcloud={"zoom_width":100}