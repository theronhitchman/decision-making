%% Fall 2013 MDM Homework Template
\documentclass[12pt,letterpaper]{article}

\usepackage[utf8]{inputenc}
\usepackage[T1]{fontenc}
\usepackage{amsmath, amsthm}
\usepackage{amsfonts}
\usepackage{amssymb}
\usepackage[left=2cm,right=2cm,top=2cm,bottom=2cm,headheight=22pt]{geometry}
\usepackage{fancyhdr}
\usepackage{setspace}
\usepackage{lastpage}
\usepackage{graphicx,subcaption}

\theoremstyle{definition}
\newtheorem{problem}{Problem}


\begin{document}

%other parameters
\setlength{\parskip}{1ex plus 0.5ex minus 0.2ex}
\setlength{\parindent}{0pt}

%header and footer parameters
\pagestyle{fancy}
\lhead{Math 1100}
\chead{Weekly Homework}
\rhead{Due: February 25}
\lfoot{}
\cfoot{\emph{Prof. Hitchman}}
\rfoot{}

\begin{center}
{
\Large
\textbf{Cantor: Assignment \#1}
}
\end{center}

On the first day of our unit, we considered the following two counting problems.

\begin{problem}
There is a meeting of 31 people. How many ways can one choose a team of two people from those attending the meeting?
\end{problem}

\begin{problem}
There is a meeting of 31 people. How many ways can one choose a team of 29 people from those attending the meeting?
\end{problem}

By now, you may know ways to solve each of these problems by counting. 
The point of this assignment is to show that \emph{only need to know how solve one of them.}

\textbf{In one written page, explain how you can see that these two problems have the same number of solutions.  Please avoid actually counting how many solutions there are to either problem.}

\section*{Specifications for Grading}

To earn credit, this assignment must
\begin{itemize}
\item be typed, of no more than one page in length;
\item address the prompt above;
\item conform to reasonable standards for grammar, spelling, and usage of the English language with minimal errors. (You may consider seeking help on writing from the Writing Center in the Academic Learning Center. http://www.uni.edu/unialc/writing-center);
\item be turned in by 2pm (the end of class) on Wednesday, February 25.
\end{itemize}




\end{document}
%sagemathcloud={"zoom_width":100}