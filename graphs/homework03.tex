%% Fall 2013 MDM Homework One
\documentclass[12pt,letterpaper]{article}

\usepackage[utf8]{inputenc}
\usepackage[T1]{fontenc}
\usepackage{amsmath}
\usepackage{amsfonts}
\usepackage{amssymb}
\usepackage{amsthm}
\usepackage[left=2cm,right=2cm,top=2cm,bottom=2cm,headheight=22pt]{geometry}
\usepackage{fancyhdr}
\usepackage{setspace}
\usepackage{lastpage}
\usepackage{url}

\theoremstyle{definition}
\newtheorem{task}{Task}

\begin{document}

%other parameters
\setlength{\parskip}{1ex plus 0.5ex minus 0.2ex}
\setlength{\parindent}{0pt}

%header and footer parameters
\pagestyle{fancy}
\lhead{Math 1100}
\chead{Weekly Homework}
\rhead{Due: Friday, January 22}
\lfoot{} 
\cfoot{\emph{Prof. Hitchman}} 
\rfoot{} 

\begin{center}
{
\Large
\textbf{Graphs: Assignment \#3}
}
\end{center}

Your paper should have the following information on it.
\begin{itemize}
\item Your name
\item Your student ID number 
\item Which section you are in: 02 MWF, or 04 TTh
\end{itemize}


\subsection*{Specifications for Grading}

To earn a passing mark, your assignment must:
\begin{itemize}
\item be typed, and at least one page and no more than two pages in length. Diagrams may be hand drawn.
\item address the tasks and questions below.
\item explain your ideas in complete sentences. Use paragraphs to organize your thoughts.
\item conform to reasonable standards for grammar, spelling, and usage of the English language with minimal errors. (You may consider seeking help on writing from the Writing Center in the Academic Learning Center. http://www.uni.edu/unialc/writing-center)
\item be turned in by 3pm on Friday, January 22.
\end{itemize}



\subsection*{What to do}

\begin{task}
We spent a lot of time studying planar graphs, and along the way we learned that $K_{3,3}$ and $K_5$ are 
not planar. Make an example of a new graph $G$ which has the following properties:
\begin{itemize}
\item The graph $G$ is not planar.
\item The graph $G$ is not equal to $K_{3,3}$.
\item The graph $G$ is not equal to $K_5$.
\end{itemize}
\end{task}

\begin{task}
The King in K\"{o}nigsberg decided the best way around his bridges problem is to build some more bridges! Draw a 
new map of K\"{o}nigsberg which includes the seven original bridges, plus some new ones that make the puzzle
actually have a solution. Then give clear directions for how a citizen of K\"{o}nigsberg could walk in a closed circuit
around the city while crossing each of the bridges, old and new, exactly once.

How many new bridges do you really \emph{need} to make an Eulerian circuit? Can you get away with only one more? only two more?
\end{task}

\begin{task}
Discuss how to turn the concrete \emph{Five Rooms Puzzle} into a task about Eulerian circuits on a certain graph. Draw pictures and explain your ideas completely.
\end{task}

\end{document}


















