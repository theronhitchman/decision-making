\documentclass{article}
\usepackage{paralist}

\begin{document}
\title{Exam Specifications\\ Graph Theory Unit}
\author{Math in Decision Making}

\maketitle

\section*{General Information}

The in-class examination for the \emph{Graph Theory} unit will be administered 
on Thursday, February 11. As described in the syllabus, the exam will be split into two separate parts: a Foundational
part, and an Advanced Part. These two parts will be done during a single class meeting.

Each of these parts will be marked on a ``Proficient/Not Proficient'' basis. The two parts will be considered 
independently.

\section*{Expectations for the Foundational Exam}

To earn a mark of  ``Proficient,'' a student must demonstrate that they can do all of these things:
\begin{compactitem}
\item Work with the concept of planarity: including deciding when a graph is planar and possibly redrawing a 
graph in a planar drawing
\item Compute basic invariants for graphs accurately in simple cases: including vertices, edges, components, degree 
sequence, cycles, colorings, and the chromatic number.
\item Decide if a graph is Eulerian and find an Eulerian cycle in the case that it is.
\end{compactitem}

\section*{Expectations for the Advanced Exam}

To earn a mark of ``Proficient,'' a student must demonstrate competence with using our complete list of invariants
to determine if a variety of graphs are the isomorphic or are not isomorphic.





\end{document}


