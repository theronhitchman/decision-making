\documentclass[12pt,letterpaper]{article}

\usepackage[utf8]{inputenc}
\usepackage[T1]{fontenc}
\usepackage{amsmath}
\usepackage{amsfonts}
\usepackage{amssymb}
\usepackage{amsthm}
\usepackage[left=2cm,right=2cm,top=2cm,bottom=2cm,headheight=22pt]{geometry}
\usepackage{fancyhdr}
\usepackage{setspace}
\usepackage{lastpage}
\usepackage{graphicx}
\usepackage{caption}
\usepackage{subcaption}
\usepackage{paralist}
\usepackage{url}

\theoremstyle{definition}
\newtheorem{question}{Question}
\newtheorem{example}{Example}
\newtheorem{exercise}[question]{Exercise}
\newtheorem*{challenge}{Challenge}
\newtheorem*{theorem}{Theorem}
\newtheorem*{definition}{Definition}

\begin{document}

%Paramètres de mise en forme des paragraphes selon les normes françaises
\setlength{\parskip}{1ex plus 0.5ex minus 0.2ex}
\setlength{\parindent}{0pt}

%Paramètres relatifs aux en-têtes et pieds de page.
\pagestyle{fancy}
\lhead{Theron J Hitchman}
\chead{\Large Reading and Guided Practice \#03}
\rhead{Spring 2016}
\lfoot{\emph{Math and Decision Making }}
\cfoot{}
\rfoot{\emph{\thepage\ of \pageref{LastPage}}}

\section*{Introduction}

We will collect some new ideas that help us distinguish graphs as non-isomorphic.  This reading has many more small exercises than usual.

\section*{Goals}
At the end of this assignment, a student should be able to:
\begin{compactitem}
\item use the terms \emph{degree}, \emph{degree sequence}, \emph{cycle}, \emph{length of a cycle}, and \emph{tree} properly.
\item discuss the concept of an invariant.
\item State and use a theorem about the sum of the degrees of the vertices in a graph.
\end{compactitem}
A student might also be able to:
\begin{compactitem}
\item Give a proof of the theorem about the sum of degrees of a graph.
\end{compactitem}

\section*{Reading and Questions for Graph Theory Meeting Four}

\subsection*{Invariants for Graphs}

We have now collected a whole list of interesting properties of graphs. Some of these properties are so integral
the the basic structure of a graph, that they are the same no matter how you draw the graph. Such a thing is
called an \emph{invariant} of the graph. What are some of these things? Let's make a list of a few we have previously discussed.
\begin{itemize}
\item The number of vertices.
\item The number of edges.
\item The number of components.
\end{itemize}

The main point of an invariant is that it helps you say for sure when two graphs are non-isomorphic. That is, it helps you say when two graphs are genuinely different. Can you see this for our examples? If two graphs are isomorphic, then the quantities above are the same for the two graphs.

So, if we have a pair of graphs and one of those quantities is \textit{different} between the two, then the graphs
cannot be isomorphic.

\begin{exercise}
Make a pair of graphs which are not isomorphic because they have a different number of vertices.
\end{exercise}

\begin{exercise}
Make a pair of graphs which have the same number of vertices, but which are not isomorphic because they 
have different numbers of edges.
\end{exercise}

\begin{exercise}
Make a pair of graphs which have the same number of vertices and the same number of edges, but which are 
not isomorphic because they have different numbers of components.
\end{exercise}

\subsection*{Degrees and the Degree Sequence}

Here is another invariant, which is at the next level of subtlety. Each vertex of a graph has a certain number of 
edges connected to it. The number of edges connected to a vertex is called the \emph{degree} of the vertex.
Our favorite graphs are distinguishable by looking at the degrees of their vertices.
\begin{figure}[h]
\centering
\begin{subfigure}[b]{.4\textwidth}
\includegraphics[width=\textwidth]{images/k5.png}
\caption{The graph $K_5$}
\label{figure:k5}
\end{subfigure}
\begin{subfigure}[b]{.4\textwidth}
\includegraphics[width=\textwidth]{images/k3,3.png}
\caption{The graph $K_{3,3}$}
\label{figure:k33}
\end{subfigure}
\caption{Our Two Best Friends}
\label{figure:complete_graphs}
\end{figure}


\begin{exercise}
Check that $K_5$, the complete graph on five vertices, has every vertex of degree equal to $4$. 
\end{exercise}

\begin{exercise}
Check that $K_{3,3}$, the complete bipartite graph on three and three vertices, has every vertex of degree $3$.
\end{exercise}

\begin{exercise}
For each vertex in the graph in Figure \ref{figure:deg-ex}, find the degree of that vertex. Label the graph by writing the degree next to each vertex.
\end{exercise}

\begin{figure}[h]
\centering
\includegraphics[width=.3\textwidth]{images/disconnected.png}
\caption{Our Disconnected Friend}
\label{figure:deg-ex}
\end{figure}

Usually, it is helpful to know ALL of the degrees of all the different vertices, so we make a list. This list is called
the \emph{degree sequence} of the graph, and is always written in \textit{decreasing order}. So, the degree
sequence of $K_5$ is $\{4,4,4,4,4\}$, and the degree sequence for $K_{3,3}$ is $\{3,3,3,3,3,3\}$. 
If two graphs are isomorphic, then they must have the same degree sequence.

\begin{exercise}
Check that the degree sequence of the disconnected graph in Figure \ref{figure:deg-ex} is $\{2,2,2,1,1,0\}$.
\end{exercise}

\begin{exercise}
Find and draw and example of a graph that has degree sequence equal to $\{2,2,2,2,1,1\}$.
\end{exercise}

\begin{exercise}
Find \textbf{another} example of a graph which has degree sequence equal to $\{2,2,2,2,1,1\}$. This graph should
be non-isomorphic to your last example. (Hint: if you last graph was connected, find a graph which is disconnected, 
or \textit{vice versa}.)
\end{exercise}

\subsection*{A Curious Fact about the Degree Sequence}

If you look at enough examples of graphs and their degree sequences, you will note a curious fact. Again, we state it as a \emph{theorem}. Is that word new to you? It is the second time I have used it. It has a special meaning to 
mathematicians.

A \emph{theorem} is a mathematical statement which has a proof. Now, don't let that frighten you. A proof is just
a careful argument that mathematicians have decided is correct and convincing. 

\begin{theorem}[Sum of Degrees Theorem]
The sum of all of the degrees of a graph is an even number.
\end{theorem}

\begin{exercise}
Check that the Sum of Degrees Theorem holds for each of the example graphs you have seen so far in this assignment.
\end{exercise}

For fun, I'll give you the first crack at this one, with a hint.

\begin{challenge}[Hint for the Sum of Degrees Theorem]
Come up with  way to explain why the Sum of Degrees Theorem is true by considering the edges instead.
\end{challenge}


\subsection*{Cycles and Trees}

An important kind of special structure in a graph is called a \emph{cycle}. A \emph{cycle} is a path along the graph that closes up, but never repeats an edge. That is, we can start from a vertex, walk along the graph by taking edges
from one vertex to the other until we end where we started, but along the way, we are not allowed to repeat an edge. We are allowed to repeat a vertex!

If we also require that we can't repeat a vertex in the middle (we of course begin and end in the same place), then
we call the cycle a \emph{simple cycle}.

\begin{exercise}
Find a cycle in $K_5$ which uses three edges.

Then find a cycle in $K_5$ that uses four edges.

Then find a cycle in $K_5$ that uses five edges.
\end{exercise}

The number of edges used in a cycle is called the \emph{length} of the cycle.

\begin{exercise}
Find a cycle of length 4 in $K_{3,3}$. 

Then find a simple cycle in $K_{3,3}$. 

Then find a cycle in $K_{3,3}$ which is NOT simple and has length 8.

Then find a cycle in $K_{3,3}$ which is NOT simple and has length 12.
\end{exercise}


\subsubsection*{How Are Cycles Helpful?}

The collection of cycles in a graph, and their lengths, are also invariants for a graph!

\begin{exercise}
Find an example of a pair of graphs that are different because one of them has a cycle of length 5 but the
other one doesn't.
\end{exercise}

\subsubsection*{One Last Word}

A graph which is connected and has no cycles is called a \emph{tree}.

\begin{exercise}
Draw a graph which is a tree and has 5 vertices.
\end{exercise}

\begin{exercise}
Draw a graph which is a tree and has 12 vertices.
\end{exercise}


\section*{Final Note}

Have you figured out how to resolve either the Five Cities Puzzle or the Three Utilities Puzzle, yet?

%\begin{thebibliography}{9}
%\end{thebibliography}

\end{document}
%sagemathcloud={"zoom_width":100}























