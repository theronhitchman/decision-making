\documentclass[12pt]{amsart}
\usepackage[margin=1in]{geometry}
\usepackage{paralist}

\theoremstyle{definition}
\newtheorem{question}{Question}

\begin{document}
\begin{center}
\textbf{\Huge
Lesson Plan: Meeting One
}
\end{center}
\vspace{.5in}

\subsection*{Materials Required:} Lots of 18'' lengths of string. Two per student.

\subsection*{The Plan}

\begin{itemize}

\item Short introduction to class: Me, three independent units, url for web page/syllabus; check up on audience: how many have this as terminal class? how many going on to another math course?

\item Planarity of $K_5$ activity. 
\begin{enumerate}
\item bring three volunteers to front. Describe idea of connecting cities with highways. Ask them to do it.
\item Note abstractions, check everyone is on board
\item divide into groups of five students. tell them task is to join the five cities so each pair of cities has a dedicated 
highway between them. Hand out strings and let them go.
\item Work the room.
\item Now give them the Five Cities Puzzle:  "Intersections and overpasses are expensive. Can you reorganize your cities and roads so that the connections are the same, but no pair of highways overlap?"
\end{enumerate}

\item At 35 minutes into class, take a 3 minute break. The activity won't be over, yet. It is okay. Collect Strings.

\item Have short class discussion about progress so far: make sure words \emph{vertex, edge, drawing, isomorphic} come up.

\item Give them the Three Utilities Puzzle. Work in pairs or triples at seats. Use pen and paper this time.
 
\item Discuss two types of homework: RGP vs weekly assigments
 
\end{itemize}


\end{document}