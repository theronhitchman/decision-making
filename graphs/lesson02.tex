\documentclass[12pt]{amsart}
\usepackage[margin=1in]{geometry}
\usepackage{paralist}

\theoremstyle{definition}
\newtheorem{question}{Question}

\begin{document}
\begin{center}
\textbf{\Huge
Graphs Lesson Plan: Meeting Two
}
\end{center}
\vspace{.5in}

\subsection*{Phase One: Peer Instruction Questions}

Introduce the idea behind Peer Instruction as a hack for the "work it out, explain it, defend it" paradigm from ed research literature. Get up to speed with basic Poll Everywhere set up. (quickly!) 

Hand out the file mdm-graphs-handout02.pdf

Do these questions: (see images in mdm-graphs-handout02.pdf)

\begin{question}
Is this a (single) graph?
\end{question}

Get idea of connectedness. Discuss word components.

\begin{question}
How many vertices does this graph have?
\end{question}

``be careful about intersections. vertices are declared in advance and indicated.''

\begin{question}
How many components does this graph have?
\end{question}

sneaky\dots

\begin{question}
How many edges does this graph have?
\end{question}

``be careful about intersections. vertices and edges are declared in advance and indicated."

\begin{question}
Is this a graph?
\end{question}

``To loop or not to loop? We will not allow loops.''

\subsection*{Short Break}

\subsection*{Phase Two}

Give a hint about the Five Cities Puzzle. ``If you pick three vertices, they must form a triangle. Try considering where you might place the other two vertices, but still keep things without crossings.''

Give them 10 minutes to work. At the 4 minute mark (or when a fraction of them seems uncomfortable and tired), 
tell them that is how long they have been working. Discuss briefly that mathematics doesn't go fast in general. Then back to work.

\subsection*{Phase Three: Euler's Formula Intro}

Have each student take out a clean piece of paper (half sheet will do). "Draw three connected graphs that definitely don't have any crossings. Then make a table like this:"

Draw an example. Then show how to make a table with $V$, $E$, $R$ and $V-E+R$. discuss counting $R$ carefully.
Give them six to seven minutes to do the work.

Collect papers.

Draw another example on the board and note that $V-E+R=2$ again. Ask class what numbers they got... Point out how weird that is.

\end{document}