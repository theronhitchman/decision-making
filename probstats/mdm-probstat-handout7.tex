\documentclass[12pt]{amsart}
\usepackage[margin=1in]{geometry}
\usepackage{paralist}

\theoremstyle{definition}
\newtheorem{task}{Task}

\begin{document}

\begin{center}
\textbf{\Huge
Chancy Business:\\ Discussion \#7
}
\end{center}


\vspace{.5in}

\section*{Aiming for a New Model}
We will denote the probability of an event $A$ with notation $\Pr(A)$.

An archer's target has four concentric circles around a bull's-eye. For a certain archer, the probabilities of scoring are as follows:
\begin{compactitem}
\item $\Pr(\text{hit the bull's-eye}) = 0.1$.
\item $\Pr(\text{hit first circle, but not bull's-eye}) = 0.3$.
\item $\Pr(\text{hit second circle, but no better}) = 0.2$.
\item $\Pr(\text{hit third circle, but no better}) = 0.2$.
\item $\Pr(\text{hit fourth circle, but no better}) = 0.1$.
\end{compactitem}
Her shots are independent events.
\vspace{1in}
\begin{compactitem}
\item[a)] What is the probability that in two shots she scores a bull's-eye on the first shot, and the third circle on her second shot?
\vspace{.5in}
\item[b)] What is the probability that in two shots she hits the bull's-eye once and the third circle once.
\vspace{.5in}
\item[c)] What is the probability that on any one shot she misses the target entirely?
\end{compactitem}

\vspace{.5in}

\subsection*{What is a good way to modify a decision tree to work in this situtation?}

\end{document}