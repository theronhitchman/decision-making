\documentclass[12pt]{amsart}
\usepackage[margin=1in]{geometry}
\usepackage{paralist}

\theoremstyle{definition}
\newtheorem{question}{Question}

\begin{document}
\begin{center}
\textbf{\Huge
Peer Instruction Questions for Third Week
}
\end{center}
\vspace{.5in}


\begin{question}
Roll a pair of (six-sided) dice. What is the probability that you roll a sum of three?
\begin{compactitem}
\item 1/11
\item 2/12
\item 1/12
\item something else
\end{compactitem}
\end{question}

The correct answer is: \textbf{something else} --- The correct probability is 2/36.


\begin{question}
Our game is to roll a pair of (six-sided) dice. What is the probability that neither of the dice you roll shows a 4?
\begin{compactitem}
\item 11/36
\item 11/12
\item 10/36
\item something else
\end{compactitem}
\end{question}

The correct answer is: \textbf{something else} --- The correct probability is 25/36.


\begin{question}
Our game is to roll a pair of (six-sided) dice. What is the probability that the sum of the numbers is something different from 4?
\begin{compactitem}
\item 10/11
\item 3/36
\item 33/36
\item something else
\end{compactitem}
\end{question}

The correct answer is:  \textbf{33/36}


\begin{question}
Our game is to roll a pair of (six-sided) dice. We do this four times. What is the probability that we get a sum of 9 at least once?
\begin{compactitem}
\item $(4/36)^4$
\item $1 - (32/36)^4$
\item $1 - (4/36)^4$
\item something else
\end{compactitem}
\end{question}

The correct answer is: $\mathbf{1 - (32/36)^4}$. This is the hardest question in this set, as far as I am concerned. The key is to use the complement rule twice!


\begin{question}
Suppose you have 6 credit cards, but your wallet only has 4 empty spaces for cards available. How many different ways can you choose cards to carry around?
\begin{compactitem}
\item $6\cdot 5 \cdot 4 \cdot 3$
\item $(6\cdot 5 \cdot 4 \cdot 3) / (4\cdot 3 \cdot 2 \cdot 1)$
\item $ 4\cdot 3 \cdot 2 \cdot 1$
\item something else
\end{compactitem}
\end{question}

The correct answer is: $\mathbf{(6\cdot 5 \cdot 4 \cdot 3) / (4\cdot 3 \cdot 2 \cdot 1)}$ --- this is a situation where you are choosing an \emph{unordered} set of 4 things out of a possible 6 things.


\begin{question}
Suppose you have 7 birthday presents that you want to arrange in your front yard for all the neighbors to see when they walk by your home. They are big enough that you can only really line them up: (jet ski, motorcycle, convertible, wax figure of Brad Pitt, A new refrigerator in the shape of ex-Chicago Bears defensive lineman William Perry, fishing boat, and a stack of 35 personalized bowling balls).
\begin{compactitem}
\item Awesome!
\item Who is William Perry?
\end{compactitem}
\end{question}

The correct answer is: \textbf{Awesome!}


\begin{question}
How many ways can you line these 7 birthday presents up for all to appreciate?
\begin{compactitem}
\item $7\cdot 6 \cdot 5 \cdot 4 \cdot 3 \cdot 2 \cdot 1$
\item $(7\cdot 6 \cdot 5 \cdot 4 \cdot 3 \cdot 2 \cdot 1)/2$
\item $ 6 \cdot 5 \cdot 4 \cdot 3 \cdot 2 \cdot 1$
\item something else
\end{compactitem}
\end{question}

The correct answer is:  $\mathbf{7\cdot 6 \cdot 5 \cdot 4 \cdot 3 \cdot 2 \cdot 1}$. This is a situation where you are choosing an \emph{ordered} sequence of 7 items out of 7 possible items.


\end{document}