%% Fall 2013 MDM Homework Template
\documentclass[12pt,letterpaper]{article}

\usepackage[utf8]{inputenc}
\usepackage[T1]{fontenc}
\usepackage{amsmath}
\usepackage{amsfonts}
\usepackage{amssymb}
\usepackage{amsthm}
\usepackage[left=2cm,right=2cm,top=2cm,bottom=2cm,headheight=22pt]{geometry}
\usepackage{fancyhdr}
\usepackage{setspace}
\usepackage{lastpage}

\theoremstyle{definition}
\newtheorem{task}{Task}

\begin{document}

%other parameters
\setlength{\parskip}{1ex plus 0.5ex minus 0.2ex}
\setlength{\parindent}{0pt}

%header and footer parameters
\pagestyle{fancy}
\lhead{Math 1100}
\chead{Weekly Homework}
\rhead{Due: 28 January}
\lfoot{} 
\cfoot{\emph{Prof. Hitchman}} 
\rfoot{} 

\begin{center}
{
\Large
\textbf{Written Assignment \#2}
}
\end{center}


\section*{Specifications for Grading}

To earn a passing mark, your assignment must:
\begin{itemize}
\item be typed, and at least one page and no more than two pages in length. Diagrams
may be hand drawn, but should be clear enough to read easily.
\item address the tasks below.
\item include a complete discussion of the puzzle and your solution, and why you know it works.
\item conform to reasonable standards for grammar, spelling, and usage of the English language with minimal errors. (You may consider seeking help on writing from the Writing Center in the Academic Learning Center. http://www.uni.edu/unialc/writing-center)
\item be turned in by 3pm on Friday, January 22.
\end{itemize}


\section*{What to Do}

\begin{task}
Draw the diagram which corresponds to this symbol for an attempted solution to PHP-2: $A^*BAB^*AB.$
\end{task}

\begin{task}
If we read the diagram from the last task in the opposite direction, what symbol do we get?
\end{task}

\begin{task}
In less than one written page, describe what the puzzle PHP3-2 is, give a solution, and describe how you know it works.
You should write as if you are describing this to a friend who is not in our class, but could be. 
(If it helps, choose someone real about whom you can think when writing.)
\end{task}


\clearpage



\end{document}