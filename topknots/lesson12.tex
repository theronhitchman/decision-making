\documentclass[12pt]{amsart}
\usepackage[margin=1in]{geometry}
\usepackage{paralist}

\theoremstyle{definition}
\newtheorem{question}{Question}

\begin{document}
\begin{center}
\textbf{\Huge
Lesson Plan: Meeting Twelve
}
\end{center}
\vspace{.5in}

\section*{Phase 1}
Do Peer Instruction polling with PollEverywhere.com to discuss linking number.

\begin{question}
Linking number is an invariant of links under isotopy. T/F
\end{question}

\begin{question}
This crossing, with orientations as shown, is a $+1$ crossing. T/F
\end{question}

\begin{question}
If you change the orientation of a link, the linking numbers change sign. T/F
\end{question}

\begin{question}
A link has as many linking numbers as it has components. T/F
\end{question}

\begin{question}
What is the linking number of the two components in this link.
\begin{compactitem}
\item $0$ 
\item $1$
\item $2$
\item $3$
\item greater than $3$
\end{compactitem}
\end{question}

\section*{Phase 2}
Discuss computing linking number. 
Do an example.

\end{document}