\documentclass{article}
\usepackage{paralist}

\begin{document}
\title{Exam Specifications\\ Topology Unit}
\author{Math in Decision Making}

\maketitle

\section*{General Information}

The in-class examination for the \emph{Topology: Picture Hanging Puzzles and Knots} unit will be administered 
on Friday, February 12. As described in the syllabus, the exam will be split into two separate parts: a Foundational
part, and an Advanced Part. These two parts will be done during a single class meeting.

Each of these parts will be marked on a ``Proficient/Not Proficient'' basis. The two parts will be considered 
independently.

\section*{Expectations for the Foundational Exam}

To earn a mark of  ``Proficient,'' a student must demonstrate that they can do all of these things:
\begin{compactitem}
\item Use our methods of study from picture hanging puzzles like PHP2, PHP3-1, and PHP3-2 to solve a new, 
related puzzle, and explain the solution.
\item Compute basic invariants for knots and links accurately: including tricolorability, number of components, and the linking number.
\item Use a sequence of Reidemeister moves to demonstrate that two knot diagrams actually represent the same knot.
\end{compactitem}

\section*{Expectations for the Advanced Exam}

To earn a mark of ``Proficient,'' a student must demonstrate competence with using our complete list of invariants
to determine if a variety of knots and links are the same or are different.





\end{document}


